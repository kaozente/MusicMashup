\section{Ausblick}

Aufgrund der begrenzten Entwicklungszeit kann das System an verschiedenen Stellen noch deutlich weiter entwickelt werden. Im Folgenden wird beschrieben, welche Entwicklungen das System in großem Maße verbessern würden.


\subsection{Asynchrone Website}
Momentan werden alle Anfragen zu Drittservern zunächst vollständig serverseitig abgewickelt, danach wird der gesamte Inhalt inklusive Empfehlungen an den Browser gesandt. Für den Nutzer entsteht so eine lange Wartezeit.
Eine Alternative dazu wäre die Verwendung von AJAX, also Asynchronous Javascript and XML. Mit dieser Technologie lassen sich nach dem Laden einer Website noch weitere Inhalte nachladen. So könnte man die Grundwebsite direkt übertragen, dann die Informationen über den Künstler nachladen und schließlich die Empfehlungen einholen. Natürlich würde die Ladezeit dadurch insgesamt nicht sinken, sondern durch den Zusatz an asynchronen Abrufen eher noch leicht steigen, die Benutzererfahrung wäre jedoch wesentlich verbessert, da die Zeit, in der nichts auf der Seite passiert, stark sinkt. Der Benutzer könnte so zum Beispiel schon den Abstract der Band lesen, während die Empfehlungen noch geladen werden. Auch könnte man nicht alle Empfehlungen direkt laden, sondern mit einer Auswahl der relevantesten beginnen und dann die weiteren auf verschiedene Seiten verteilen (\glqq Pagination\grqq).


\subsection{Abstraktion schaffen}


Website-Logik, das Abrufen der Daten von verschiedenen Endpoints, das Cachen von Daten und das Finden der Empfehlungen geschehen derzeit in einer einzigen Klasse. Um die Qualität und die Wartbarkeit des Codes zu erhöhen ist es sinnvoll, das System zu modularsieren und entsprechende Teile des Systems in einzelne Klassen auszulagern.


\subsection{Aufsetzen eines eigenen SPARQL-Endpoints}


Das Aufsetzen eines eigenen SPARQL-Endpoints würde die Handhabung des Cachings vereinfachen und zusätzlich die Möglichkeit bieten, anderen Anwendungen einfachen Zugang zu den von uns gesammelten Daten zu schaffen. In der aktuellen Implementierung schreibt der Parser alle Informationen in Turtle-Dateien. Das heißt, dass erstens bei Abruf vieler Artists auch viele Dateien enstehen und zweitens ist Turtle ein serialisierte Form von Linked Data, die im Vergleich zu einem SPARQL-Endpoint schlechter zum Abruf von Daten geeignet ist. 


\subsection{Personalisierung}
Bisher wird als Eingabe vom Nutzer nur die eingegebene Ausgangs-Band ausgewertet. Dies könnte man erweitern, zum einen durch Auswerten von statistischen Daten, etwa welche Empfehlungen für welche Band besonders oft geklickt werden und daher möglicherweise besonders relevant sind, zum anderen durch Berücksichtigung weiterer direkter Eingaben. So könnte man beispielsweise den Standort des Nutzers bei der Anzeige von Konzerten berücksichtigen. Eine weitere Möglichkeit wäre die Einbindung der Facebook-API\footnote{\url{https://developers.facebook.com/}}. So könnte zum Beispiel für einen Nutzer von Facebook die Menge der von ihm mit einem \glqq Like\grqq \ versehenen Künstler geladen werden und als Grundlage für Empfehlungen herangezogen werden.

Auch die Anzeige der Ergebnisse ließe sich erweitern, indem die Liste der Empfehlungen nachträglich durch weitere Angaben gefiltert werden könnte, zum Beispiel nach einem bestimmten Genre. Auf Dauer könnte dies sogar pro Nutzer gespeichert und ausgewertet werden, sodass die Sortierung durch den Voting-Alghoritmus auch den persönlichen Geschmack des Nutzers berücksichtigt.

Ein weiterer großer Schritt wäre auch die Einbindung von Benutzeraccounts. Dies würde Benutzern zum Beispiel die Möglichkeit geben neu entdeckte Bands zu markieren und zu speichern, um einen erneuten Zugriff zu erleichtern. Ausgehend davon wäre sogar eine Anbindung an soziale Netzwerke denkbar, um neu entdeckte Künstler mit sozialen Kontakten zu teilen.