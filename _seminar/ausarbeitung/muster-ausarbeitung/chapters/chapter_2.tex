%
\section{Aufbau und Inhalt der Seminararbeit}
\label{sec_aufbau}

Im vorangegangenen Kapitel hatten wir bereits die Gliederung einer Seminararbeit kurz vorgestellt und erläutert, welche inhaltlichen Punkte in der \glqq Einleitung\grqq\, behandelt werden sollten.
Die folgenden Abschnitte skizzieren inhaltlich die übrigen der bereits genannten Gliederungspunkte.

\subsection{Verwandte Arbeiten und wissenschaftlicher Hintergrund (Related Work)}
%%
Hier sind vor allem zwei inhaltliche Punkte zu berücksichtigen:
\begin{itemize}
\item {\bf Notwendige Vorarbeiten und Grundlagen, die zum Verständnis der Arbeit notwendig sind}

Keine bzw. kaum eine Arbeit beginnt als \glqq tabula rasa\grqq , d.h. meist bauen wir auf  vorhandenen Grundlagen bzw. Vorarbeiten auf.
Die zum Verständnis der eigenen Arbeit notwendigen Grundlagen und Voraussetzungen müssen in diesem Kapitel skizziert bzw. zusammengefasst werden.
Dabei sollte man vom durchschnittlichen Kenntnisstand eines Informatikers ausgehen, d.h. Allgemeinplätze und allzu Grundlegendes hat hier nichts zu suchen.
Genauso soll hier nicht notwendigerweise eine kompletter Wissenschaftszweig in epischer Tiefe ausgebreitet werden, sondern lediglich die zum Verständnis notwendigen Teilbereiche in skizzenhafter Form und mit Angabe von Literaturhinweisen zusammengefasst werden. (Zum Beispiel können hier die Grundlagen und Vorzüge von Linked Open Data erläutert werden.)

\smallskip

\item {\bf Alternative Ansätze und ggfs. Forschungsarbeiten zum Thema}
Besonders wichtig ist es, spezielle Vorarbeiten und alternative Ansätze zum behandelten Thema darzulegen.
Gibt es zu der von Ihnen gewählten Problemstellung alternative Lösungen, die einen anderen oder vergleichbaren Ansatz verfolgen? Wie unterscheiden sich diese Lösungen von Ihrem Vorschlag, wo liegen Vor- und Nachteile des jeweiligen Ansatzes?

Wichtig ist, dass Sie jede der vorgestellten Arbeiten 
\begin{itemize}
\item korrekt zitieren (Bibliografie),
\item kurz die wichtigsten Ergebnisse bzw. Strategien skizzieren und
\item diese (kurz und knapp) in Zusammenhang mit ihrer eigenen Arbeit stellen. 
\end{itemize}
Wie unterscheidet sich der eigene Ansatz von den vorgestellten Arbeiten? 
Warum ist der eigene Ansatz eventuell erfolgsversprechender? 

\end{itemize}

\subsection{Eigener Ansatz zur Lösung der gestellten Aufgabe}
%%
Hier haben Sie die Freiheit, Ihren eigenen Arbeiten angemessen viel Raum zur Verfügung zu stellen.
Achten Sie dabei auf einen logischen Aufbau der Darstellung, d.h. Grundlegendes zuerst.
\begin{itemize}
\item Wie sind Sie vorgegangen?
\item Wo gibt es Probleme?
\item Wie werden diese gelöst?
\item Schreiben Sie in verständlicher Weise und drücken Sie sich dabei jeweils möglichst präzise, d.h. unmissverständlich aus (vgl. Kap.~\ref{sec_stil})
\item Verwenden Sie Abbildungen, Tabellen und Beispiele.
\item Setzen Sie kein Wissen als implizit vorhanden voraus, sondern sprechen Sie explizit alle Probleme und wichtigen Fakten an.
\item Wichtig: Was Sie hier nicht beschreiben, können wir nicht bewerten!
\end{itemize}
Bedenken Sie dabei stets, dass ein Leser nicht dasselbe Wissen besitzen kann wie Sie und das Sie ihm deshalb ihre Ergebnisse erklären müssen.


\subsection{Diskussion der erzielten Ergebnisse}
%%
In diesem Kapitel sollten Sie Ihre Ergebnisse präsentieren.
Dabei sollten (falls jeweils zutreffend) folgende Fragen beantwortet werden:
\begin{itemize}
\item Was wurde erreicht, was kann noch verbessert werden bzw. wo gibt es noch offene (evtl. aus Zeitgründen nicht implementierte) Punkte?
\item Warum ist der eigene Ansatz besser/schlechter als die zum Vergleich herangezogenen?
\item Was haben Sie aus dem Seminar mitgenommen (z.B. Wo liegen die Vorteile von Linked Open Data?)
\end{itemize}


\subsection{Zusammenfassung und Ausblick}

In diesem Abschnitt sollten die erzielten Ergebnisse noch einmal kurz zusammengefasst werden und ein Ausblick auf weiterführende Entwicklungsarbeiten gegeben werden (vgl. Kap.~\ref{sec_conclusion}). Hier können Sie z.B. ausführen, welche Arbeiten Sie aus Zeitgründen nicht umsetzen konnten, aber für wichtig oder sinnvoll erachten.