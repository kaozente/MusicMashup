% Vierte Seite = Hier geht's eigentlich richtig los
\section{Einleitung}
\label{sec_einleitung}

\noindent
Mit den vorliegenden Hinweisen versuchen wir Ihnen einen Leitfaden zum Erstellen von Seminararbeiten im Fach Informatik an die Hand zu geben.
Wie Sie sicher schon beim Lesen wissenschaftlicher Arbeiten bemerkt haben werden, folgen diese meist einem einheitlichen Aufbau.
Dies liegt nicht daran, dass die Autoren sich keine Mühe geben würden bzw. große Langweiler sind, denen eben nichts Neues einfallen würde.
Nein, ein einheitlicher Aufbau erleichtert dem Leser -- der meist nie besonders viel Zeit hat bzw. investieren möchte -- die wesentlichen Beiträge der Arbeit schnell und effizient zu erfassen.


\subsection{Die Gliederung}
Aber wie gliedert man eine wissenschaftliche Arbeit?
Meist kommt dabei das folgende einfache Schema zum Einsatz:\index{Aufbau der Arbeit}
\begin{enumerate}
\item Einleitung
\item Verwandte Arbeiten und wissenschaftlicher Hintergrund (Related Work)
\item Eigener Ansatz zur Lösung der gestellten Aufgabe -- dies können gerne mehrere Kapitel werden...
\item Diskussion der erzielten Ergebnisse
\item Zusammenfassung und Ausblick 
\end{enumerate}

Auf die Eigenheiten der einzelnen Unterpunkte werden wir im Folgenden noch genauer eingehen.
Beginnen wir einfach mit der Einleitung.\index{Einleitung}

\subsection{Inhalt der Einleitung}
Die Einleitung soll den Leser zum {\bf Thema\index{Thema} hinführen}, die Arbeit in einen Gesamtzusammenhang einordnen und einen kurzen Überblick über den Inhalt der Arbeit geben. 
Dabei sind die folgenden Punkte besonders wichtig:

\begin{itemize}
\item Motivation\index{Motivation} des Themas -- warum ist das Thema überhaupt von Bedeutung?
\item Wie ordnet sich das Thema in einen größeren Gesamtzusammenhang (z.B. den Rahmen des Seminars) ein?
\item Darlegung der grundlegenden Aufgabe: Worum geht es eigentlich? Was sind die zentralen Fragestellungen? Wie beabsichtigen wir diese zu lösen?
\item Warum lohnt sich das Weiterlesen? 
\end{itemize}
Wichtig ist, dass die Einleitung die {\bf Dramaturgie} der Arbeit quasi wie einen \glqq roten Faden\grqq\, sichtbar werden lässt.

\bigskip

Am Ende der Einleitung sollte ein \textbf{kurzer Überblick über den Inhalt} der einzelnen Kapitel folgen. 
Die vorliegende Arbeit könnte wie folgt skizziert werden:

\bigskip

Kapitel~\ref{sec_aufbau} gibt Hinweise zum Aufbau einer Seminararbeit und wie deren Inhalte zu gestalten sind.
Kapitel~\ref{sec_stil} gibt allgemeine Hinweise zur Formatierung von wissenschaftlichen Arbeiten.
In Kapitel~\ref{sec_inhalt} werden die einzelnen inhaltlichen Bestandteile der wissenschaftlichen Arbeit dargestellt, worauf in Kapitel~\ref{sec_literatur} wichtige Hinweise zur korrekten Zitierweise gegeben werden.
Kapitel~\ref{sec_conclusion} beschließt die Arbeit mit einer Zusammenfassung der Ergebnisse und einem Ausblick auf die weitere Entwicklung des eigentlichen Themas.


