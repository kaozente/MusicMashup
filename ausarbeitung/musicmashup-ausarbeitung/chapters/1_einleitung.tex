\section{Einleitung}

Das Internet ist eine Quelle für alle Arten von Daten -- seien es Enzyklopädien wie die Wikipedia oder spezielle Datenbanken für Filme, Bücher, Forschungsergebnisse und viele weitere.


Versucht man nun ein Mashup zu entwickeln, welches verschiedene Datenquellen nutzt, entstehen oftmals Probleme.

\begin{itemize}
\item Die verwendeten Datenquellen verwenden meistens unterschiedliche Formate. Das heißt, dass vor der Verwendung der Daten, diese auf ein gemeinsames Format geführt werden müssen.
\item Auch die Abfrageweise der Daten ist nicht einheitlich. Wird zum Beispiel mit einer SQL-Datenbank und einer API gearbeitet, sind für beide Datenquellen jeweils unterschiedliche Abfrageweise notwendig.
\item Die meisten Datenquellen enthalten keine Verweise auf andere thematisch relevanten Datenquellen. Somit ist ein zusätzliches Maß an Recherche erforderlich, um ausgehend von einer Datenquelle eine weitere zu finden.
\item Viele Datenquellen werden privat verwaltet und sind nicht öffentlich und somit auch nicht frei nutzbar. Solche Datenquellen können nicht verwendet werden und stellen für die Öffentlichkeit keinen Mehrwert dar.
\end{itemize}

Um diese Probleme zu lösen wurde das Prinzip von Linked Open Data entwickelt.

\paragraph{} In Kapitel 2 wird zunächst ein kurzer Überblick über die Idee hinter Linked (Open) Data gegeben und die Funktionsweise grob skizziert. Darüber hinaus wird die Funktionsweise anderer Musik-Recommender erklärt und ein Vergleich zu unserem Ansatz gezogen.

\paragraph{} In Kapitel 3 wird der von uns verfolgte Ansatz im Detail erläutert und sowohl von konzeptioneller als auch von technischer Seite beleuchtet.

\paragraph{} In Kapitel 4 kommt es zur Darstellung und Diskussion der erzielten Ergebnisse und es wird auf die Stärken und Schwächen des Systems eingegangen.

\paragraph{} Kapitel 5 ist eine Zusammenfassung der erzielten Ergebnisse und Erkenntnisse. Es folgt ein Ausblick auf mögliche Verbesserungen und Erweiterungen des Systems.